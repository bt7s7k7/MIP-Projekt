% Metódy inžinierskej práce

\documentclass[10pt,twoside,slovak,a4paper]{article}

\usepackage[slovak]{babel}
%\usepackage[T1]{fontenc}
\usepackage[IL2]{fontenc} % lepšia sadzba písmena Ľ než v T1
\usepackage[utf8]{inputenc}
\usepackage{graphicx}
\usepackage{url} % príkaz \url na formátovanie URL
\usepackage{hyperref} % odkazy v texte budú aktívne (pri niektorých triedach dokumentov spôsobuje posun textu)

\usepackage{cite}
%\usepackage{times}

\pagestyle{headings}

\title{Gamifikácia pre učenie cudzích jazykov\thanks{Semestrálny projekt v predmete Metódy inžinierskej práce, ak. rok 2022/23, vedenie: Ing. Fedor Lehocki, PhD.}} % meno a priezvisko vyučujúceho na cvičeniach

\author{Branislav Trstenský\\[2pt]
	{\small Slovenská technická univerzita v Bratislave}\\
	{\small Fakulta informatiky a informačných technológií}\\
	{\small \texttt{xtrstenskyb@stuba.sk}}
	}

\date{\small 30. september 2022} % upravte



\begin{document}

\maketitle

\begin{abstract}

Vložiť abstrakt.

\end{abstract}


\section{Úvod}

Ako napredujú technológie a svet sa stáva prepojenejším, potreba znalosti cudzích jazykov stúpa. Či už ide o zamestnanie v zahraničí, alebo prístup k širšiemu výberu informácií. Učenie cudzích jazykov však vyžaduje časté opakovanie a upevňovanie znalostí, ktoré môže byť ťažké pre študenta dlhodobo udržať. 

Gamifikácia poskytuje zdroj motivácie a pomáha udržiavať pravidelné štúdium. Príkladom sú získavanie bodov za skúšky, zábavné minihry alebo notifikácie, ktoré pripomínajú k pokračovaniu výučby. 

Tento článok vysvetlí gamifikáciu všeobecne (časť \ref{gamifikacia}), predstaví spôsoby ako prvky gamifikácie môžu pomôcť pri štúdiu cudzích jazykov (časť \ref{prvky}) a následne uvedie a popíše konkrétne aplikácie, ktoré využívajú princípi gamifikácie v praxy (časť \ref{prax}).

\section{Gamifikácia} \label{gamifikacia}

Okrem zábavných prvkov, hry umožňujú prostredníctvom svojho priebehu rozvíjanie viacerých kognitívnych schopností, ako napríklad logiku, motorické schopnosti, priestorovú orientáciu alebo jazykové schopnosti. 

Vo väčšine prípadov, proces učenia prebieha intuitívne a spontánne počas hrania. Aj keď učenie nie je hlavným cieľom hry, tréning schopností je výsledkom zapájania sa do úloh, ktoré hra vyžaduje, procesu pokus-omyl a ich neustálemu opakovaniu v rámci hernej slučky. Rozvíjanie schopností je ďalej motivované odmenami, alebo len pocitu prekonania výziev. \cite{Rego}

Gamifikácia je proces, pri ktorom sa prvky hier, považované tradične ako zábavné, používajú priamo na podporu učenia, angažovania a zručností riešenia problémov. Tieto prvky tvoria systém, v ktorom účastníci prekonávajú výzvy, ktoré obsahujú pravidlá, interaktivitu a spätnú väzbu.

Konkrétne aspekty učenia, ktoré sú vylepšené gamifikáciou sú napriklad spolupráca, participaívny prísup, samoriadené štúdium, motivácia na dokončenie zadaní alebo zjednodušenie hodnotenia. \cite{Kapp}

\section{Prvky gamifikácie} \label{prvky}

Existuje viacero spôsobov ako kategorizovať jednotlivé prvky gamifikácie. Na účely tohto článku sa použije rozdelenie na [...] .

\subsection{Ciele}

Je dôležité aby ciele hry sústredili priamo na učenie, inak existuje šanca že študent dokončí hru bez konkrétnych učebných výsledkov. Ďalej je potrebné vytvoriť prechodné ciele, ktoré sa splnia na ceste k hlavným cieľom. Vďaka ním je študent motivovaný pokračovať v hre a je mu poskytnutá spätná väzba o jeho zručnostiach.

\subsection{Mechaniky}

Mechaniký sú tvorené pravidlami, ktoré určujú priebeh hry. Diktujú ako hra funguje, formalizujú štruktúru ktorá podporuje funkcionalitu hry a riadia proces učenia počas hry. V prípade, že má hra sociálne prvky, sú potrebné aj pravidlá na kontrolu kontaktu medzi hráčmi. \cite{Rego}

\subsection{Estetika}

Estetika je dôležitá na udržanie záujmu hráča. Cieľom estetiky je vytvoriť prostredie, ktoré hráča ponorí do hry. Hlavnými faktormi estetiky sú vhodné a vzájomne zrovnané grafické prvky, pozornosť k detailom \cite{Kapp}. a vyvarovanie sa rušivým prvkom. Dôležité je brať do úvahy cieľovú skupinu, pre ktorú je hra určená, ako napríklad ich vekové kategórie. \cite{Rego}

\subsection{Herné rozmýšlanie}

Herné rozmýšľanie je premena tradičných učebných aktivít na herné aktivity. Spočíva v pozorovaní výučbových situácií a intifikovať prvky, ktoré je možné obohatiť gamifikáciou. Následne vytvoriť systémy, ktoré motivujú motivujú toto želané správanie. \cite{Werbach+Hunter}

\subsection{Spolupráca}

Aj keď gamifikované zážitky, zamerané na samoštúdium sú dostačujúce, spolupráca môže zlepšiť ich účinnosť. Cez výmenu skúseností a vzájomnú pomoc medzi študentmi sa viac zapájajú do do procesu výučby. Skupinová práca tiež vytvára záväzok, ktorý zabraňuje vyhýbaniu sa. \cite{Rego}

\subsection{Odmeny}

Systémy odmeňovania ako napríklad skóre, hviezdy alebo odznaky sú dôležité pre motiváciu. Aj keď sú odmeny virtuálne, aj tak možnosť niečo vyhrať motivuje hráča pokračovať. Tieto benefity sú zvýšené v sociálnom kontexte, kde jednotlivý študenti môže súperiť medzi sebou. Súperenie je možné využiť prostredníctvom jednoduchých herných prvkov ako zoraďovanie hráčov podľa skóre, kde sa hráč môže sústrediť na zlepšenie svojho miesta. Toto poradie tiež ponúka hráčovi spätnú väzbu. \cite{Kapp}

\subsection{Spätná väzba}

V hrách je spätná väzba, na rozdiel od tradičnej výučby, konštantne k dispozícií. Spätnú väzbu môžeme deliť na informačnú, ktorá ukazuje študentovi stupeň úspešnosti alebo chyby a edukačnú, ktorá mu pomáha k lepšiemu výkonu. Spätná väzba by mala byť prezentovaná tak aby sa o ňu hráč zaujímal. Hodnotenie by malo byť tiež opakovateľné, aby ho mal hráč možnosť zlepšiť.  \cite{Kapp}

\subsection{Postup cez úrovne}

Postup cez má tri úlohy:
\begin{enumerate}
	\item Zmenu častí priebehu hry, aby bola udržaná pozornosť hráča
	\item Zvýšenie náročností úloh, na pokračovanie vývynu zručností
	\item Tvorenie motivácie cez pocit prekonania výzvy
\end{enumerate}
Prostredníctvom úrovní si hry môžu udržať konzistentnú obtiažnosť vzhľadom na zlepšené zručnosti hráča a predísť tomu aby sa hra stala príliš jednotvárna a únavná. \cite{Kapp}

\section{Príklady gamifikácie výučby jazyka v praxy} \label{prax}

\section{Záver} \label{zaver} % prípadne iný variant názvu
Záver


%\acknowledgement{Ak niekomu chcete poďakovať\ldots}


% týmto sa generuje zoznam literatúry z obsahu súboru literatura.bib podľa toho, na čo sa v článku odkazujete
\bibliography{literatura}
\bibliographystyle{plain} % prípadne alpha, abbrv alebo hociktorý iný
\end{document}
