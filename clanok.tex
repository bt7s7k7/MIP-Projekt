% Metódy inžinierskej práce

\documentclass[10pt,twoside,slovak,a4paper]{article}

\usepackage[slovak]{babel}
%\usepackage[T1]{fontenc}
\usepackage[IL2]{fontenc} % lepšia sadzba písmena Ľ než v T1
\usepackage[utf8]{inputenc}
\usepackage{graphicx}
\usepackage{url} % príkaz \url na formátovanie URL
\usepackage{hyperref} % odkazy v texte budú aktívne (pri niektorých triedach dokumentov spôsobuje posun textu)

\usepackage{cite}
%\usepackage{times}

\pagestyle{headings}

\title{Názov\thanks{Semestrálny projekt v predmete Metódy inžinierskej práce, ak. rok 2022/23, vedenie: Ing. Fedor Lehocki, PhD.}} % meno a priezvisko vyučujúceho na cvičeniach

\author{Branislav Trstenský\\[2pt]
	{\small Slovenská technická univerzita v Bratislave}\\
	{\small Fakulta informatiky a informačných technológií}\\
	{\small \texttt{xtrstenskyb@stuba.sk}}
	}

\date{\small 30. september 2022} % upravte



\begin{document}

\maketitle

\begin{abstract}

Vložiť abstrakt.

\end{abstract}


\section{Úvod}

Motivujte čitateľa a vysvetlite, o čom píšete. Úvod sa väčšinou nedelí na časti.

Uveďte explicitne štruktúru článku. Tu je nejaký príklad.





\section{Záver} \label{zaver} % prípadne iný variant názvu
Záver


%\acknowledgement{Ak niekomu chcete poďakovať\ldots}


% týmto sa generuje zoznam literatúry z obsahu súboru literatura.bib podľa toho, na čo sa v článku odkazujete
%\bibliography{literatura}
%\bibliographystyle{plain} % prípadne alpha, abbrv alebo hociktorý iný
\end{document}
